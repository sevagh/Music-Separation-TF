%

% File final_report.tex
%
%% Based on the style files for ACL 2020, which were
%% Based on the style files for ACL 2018, NAACL 2018/19, which were
%% Based on the style files for ACL-2015, with some improvements
%%  taken from the NAACL-2016 style
%% Based on the style files for ACL-2014, which were, in turn,
%% based on ACL-2013, ACL-2012, ACL-2011, ACL-2010, ACL-IJCNLP-2009,
%% EACL-2009, IJCNLP-2008...
%% Based on the style files for EACL 2006 by 
%%e.agirre@ehu.es or Sergi.Balari@uab.es
%% and that of ACL 08 by Joakim Nivre and Noah Smith

\documentclass[12pt,letter]{article}
\usepackage[hyperref]{acl2020}
\usepackage{times}
\usepackage{latexsym}
\usepackage{amsmath}
\usepackage{fancyvrb}
\usepackage{subcaption}
\usepackage{makecell}
\usepackage{placeins}
\usepackage{hyperref}
\usepackage{graphicx}
\graphicspath{ {./} }
\renewcommand{\UrlFont}{\ttfamily\small}
\usepackage[
    %backend=biber, 
    natbib=true,
    style=numeric,
    sorting=none
]{biblatex}
\addbibresource{citations.bib}

% This is not strictly necessary, and may be commented out,
% but it will improve the layout of the manuscript,
% and will typically save some space.
\usepackage{microtype}
\usepackage[T1]{fontenc}

\aclfinalcopy % Uncomment this line for the final submission
%\def\aclpaperid{***} %  Enter the acl Paper ID here

%\setlength\titlebox{5cm}
% You can expand the titlebox if you need extra space
% to show all the authors. Please do not make the titlebox
% smaller than 5cm (the original size); we will check this
% in the camera-ready version and ask you to change it back.

\newcommand\BibTeX{B\textsc{ib}\TeX}

\newenvironment{tight_enumerate}{
\begin{enumerate}
\setlength{\itemsep}{0pt}
\setlength{\parskip}{0pt}
}{\end{enumerate}}

\newenvironment{tight_itemize}{
\begin{itemize}
\setlength{\itemsep}{0pt}
\setlength{\parskip}{0pt}
}{\end{itemize}}

\title{Music Source Separation With Different Time-Frequency Representations}

\author{Sevag Hanssian \\
  McGill University \\
 \texttt{sevag.hanssian@mail.mcgill.ca} \\
 \texttt{sevagh@protonmail.com}}

\date{}

\begin{document}
\maketitle
%%%%%%%%%%%%%%
% ABSTRACT			%
%%%%%%%%%%%%%%
\begin{abstract}
	\citet{fitzgerald1} presented an algorithm for separating a musical mix into harmonic and percussive components with the short-time Fourier transform. \citet{fitzgerald2} introduced the constant-Q transform to also separate the vocal component. \citet{driedger} created an iterative variant using a large window for the harmonic separation and a small window for the percussive separation. The field of wavelets has produced similar algorithms for tonal/transient separation, such as \citet{tfjigsaw}'s Jigsaw Puzzle. A survey of different time-frequency representations used for harmonic/percussive/vocal source separation is proposed. They will be implemented and evaluated in MATLAB, and the best performers will be combined in a hybrid algorithm to compare against a state-of-the-art neural network.
\end{abstract}

%%%%%%%%%%%%%%
% INTRODUCTION		%
%%%%%%%%%%%%%%
\section{Introduction}
\label{sec:intro}

Based on the insight that harmonic sounds exhibit spectral sparseness and temporal smoothness (narrowband and steady), and percussive sounds exhibit spectral smoothness and temporal sparseness (transient and broadband)

In source separation, a mixed song is split into its constituent components, or sources. There are different forms


%%%%%%%%%%%%%%
% RELATED WORK		%
%%%%%%%%%%%%%%
\section{Related Work}
\label{sec:related}

wowowow

\subsection{cool}

wowowow

\subsection{Code Availability}

sup bruh

\vfill
\clearpage % force a page break before references

\nocite{*}
\printbibheading[title={References}]
\printbibliography[heading=none]

\end{document}
